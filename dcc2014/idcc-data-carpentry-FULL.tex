%\documentclass[practice]{ijdc-v9}
\documentclass[15]{idcc}


\title[Data Carpentry]{Data Carpentry: \\workshops to increase data literacy for researchers}
\author{Tracy~K.~Teal}
\affil{Michigan State University, East Lansing, MI, USA}
\author{Karen~A.~Cranston}
\affil{National Evolutionary Synthesis Center (NESCent), Durham, NC, USA}
\author{Hilmar~Lapp}
\affil{National Evolutionary Synthesis Center (NESCent), Durham, NC, USA}
\author{Ethan~White}
\affil{Utah State University, Logan, UT, USA}
\author{Greg Wilson}
\affil{Software Carpentry Foundation, Toronto, Canada}
\author{Karthik Ram}
\affil{Section of Evolution and Ecology, University of California, Davis, CA, USA}
\author{Aleksandra Pawlik}
\affil{University of Manchester, United Kingdom}
\correspondence{Aleksandra Pawlik, Room 1.17 Kilburn Building, Oxford Road, University of Manchester, M13 9PL, Manchester, United Kingdom. Email: \email{aleksandra.pawlik@manchester.ac.uk} }


\begin{document}

\maketitle



\section{Abstract}
In many domains the rapid generation of large amounts of data is fundamentally changing how research is done. The deluge of data presents great opportunities, but also many challenges in managing, analyzing and sharing data. Good training resources for researchers looking to develop skills that will enable them to be more effective and productive researchers are scarce. To address this need we have developed an introductory two-day intensive workshop, Data Carpentry, designed to teach basic concepts, skills, and tools for working more effectively and reproducibly with data.\\

A survey of researchers in the National Science Foundation's BIO Centers revealed not only gaps in knowledge in data management and analysis, but also that researchers are frustrated with 
their current data workflows and know their research capacity is being limited by this lack of knowledge. In a Bioinformatics Resource Australia EMBL 2013 Community Survey Report the most emphatic outcome was the overwhelming demand for training\footnote{\url{http://braembl.org.au/news/braembl-community-survey-report-2013}}. More than 60\% of researchers surveyed said that their greatest need was additional training, compared to a meagre 5\% who need access to additional compute power.  While this survey is focused on biology and bioinformatics, the sentiment is shared by researchers in many domains and regions and has been clearly identified by ELIXIR-UK as well. Fundamentally this lack of skills and of confidence is limiting research progress. Researchers know they want to learn more and increase their data literacy but there are limited opportunities for training. \\

% Any references from ELIXIR UK on training needs?

Software Carpentry, two-day hands-on bootcamp style workshops teaching best practices in software development, have demonstrated the success of short workshops to teach foundational research skills. This model has been adapted for Data Carpentry with the objective of teaching skills to researchers to enable them to retrieve, view, manipulate, analyze and store their and other's data in an open and reproducible way, through the data lifecycle and be able to extract knowledge from data.\\

To attain this objective, we identified the following teaching subjects. 
\begin{itemize}
\item How to use spreadsheet programs (such as Excel) more effectively, and the limitations of such programs. 
\item Getting data out of spreadsheets and into more powerful tools --- using R or Python. 
\item Using databases, including managing and querying data in SQL. 
\item Workflows and automating repetitive tasks, in particular using the command line shell and shell scripts. 
\end{itemize}

In addition to the above subjects, the following skills emerged as particularly important to impart from our discussions about designing the course:
\begin{itemize} 
\item Preparing data for analysis. 
\item Using data and computational resources, in particular publicly available ones such as Amazon Web Services 
\item Conducting data and computation-heavy research more reproducibly and openly. 
\end{itemize}

Additionally, criteria has been developed to make the training most effective for researchers new to data management and analysis practices, based on research on how people 
learn best and experience through Software Carpentry workshops. 

\begin{itemize}
\item Workshops are domain specific. Each field has its own data types, analysis packages and standard problems to address. Being able to teach people in their domain let\'s them both more immediately understand the questions and approaches, and then be able to apply it to their own work. Using examples that are 'real world' to a given domain is fundamentally motivating for the skills that are being taught.

\item Workshops are a narrative that show the data lifecycle for a given dataset or problem. All components of the data lifecycle are fundamental in the quality of the final analysis. Emphasizing all the components from setting up data tables, to viewing, manipulating, analyzing, visualizing and sharing data is crucial for accurate outcomes and reproducible research. Also, this lifecycle again models a users' workflow, allowing learners to put the process in to action with their own data sets.

\item Workshops are designed for people with no prior computational experience. Learners can walk in with any level of background, but these workshops assume no prior knowledge. In this way learners should not self-select whether or not they should attend, and there is clear expectation for the pace of instruction. We also can meet researchers where they are and build on existing practices and knowledge.

\item These workshops can be focused on any research domain, not just science. Social scientists, digital humanists, librarians, and museum collections are also facing the same challenges with the digital data deluge. The same principles in the data lifecycle can be applied in any domain of research and materials adapted to meet the specific needs of that domain.

\item Running an effective workshop means having instructors trained in how to teach, particularly in a workshop format. Software Carpentry has developed an effective train-the-trainers program based on pedagogy and experience. Data Carpentry instructors will also be trained in this way with Data Carpentry specific modules developed. 

\end{itemize}

Four Data Carpentry workshops in biology have now been taught with positive response and survey assessment results demonstrating that learning objectives are being met. The research community has been enthusiastic about hosting, teaching or taking these workshops as well as been engaged in the development of materials in other domains and in expanding topics. Work is already progressing on materials for genomics, neuroscience, social science and geoscience and is expanding to include lessons on data visualization and introductory statistics.

Data Carpentry workshops won't be able to teach researchers all the skills they need in two days, but we've shown that they are a way to get the process started and that they can lower the activation energy required for researchers to be able to do more and more effective work with their data and enable research progress.


\end{document}.
